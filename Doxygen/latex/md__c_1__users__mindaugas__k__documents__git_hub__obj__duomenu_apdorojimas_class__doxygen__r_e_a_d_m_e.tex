
\begin{DoxyEnumerate}
\item Clone\textquotesingle{}inam repositoriją {\ttfamily \$ git clone \href{https://github.com/gitguuddd/Obj_Duomenu_apdorojimas.git}{\texttt{ https\+://github.\+com/gitguuddd/\+Obj\+\_\+\+Duomenu\+\_\+apdorojimas.\+git}}}
\item Compile\textquotesingle{}inam programą paleisdami C\+Make\+Lists.\+txt failą
\item Norint nuskaityti informaciją nuo failo reikia sukurti kursiokai.\+txt failą su tokia struktūra 
\begin{DoxyCode}{0}
\DoxyCodeLine{Pavarde     Vardas      ND1  ND2  ND3  ND4  ND5  ND6  ND7  ND8  Egzaminas}
\DoxyCodeLine{Pavarde1    Vardas1     10   9    4    6    4    5    0    4    7}
\DoxyCodeLine{Pavarde2    Vardas2     10   0    6    7    9    4    2    5    3}
\DoxyCodeLine{Pavarde3    Vardas3     10   0    3    b    5    2    1    4    5}
\end{DoxyCode}
 \subsection*{-\/ Paleidžiame programą }
\end{DoxyEnumerate}

\subsection*{Naudojimosi instrukcija}


\begin{DoxyItemize}
\item Paleidųs programą programa paklaus vartotojo kokiu režimu jis nori vykdyti programą\+: išsamiu ar konkrečiu
\item Išsamus režimas pasižymi labai didele pasirinkimo galimybe -\/ kiekvienam generuojamam studentu failui galima priskirti skirtingą skaičių namų darbų, galutinio pažymio skaičiavimo būdą. Šis režimas yra apkrautas vartotojo įvestimi
\item Konkretus režimas kiek galima labiau sumažina vartotojo įvesčių kiekį -\/ generuojant failus reikia tik vieną kartą įvesti namų darbų skaičių ir galutinio pažymio skaičiavimo būdą
\item Paleidųs programą ir pasirinkųs režimą bus matomas toks vaizdas 
\begin{DoxyCode}{0}
\DoxyCodeLine{Saugomi 0 studentu(o) duomenys, pasirinkite ka daryti toliau:}
\DoxyCodeLine{1. Ivesti studenta paciam}
\DoxyCodeLine{2. Generuoti studenta}
\DoxyCodeLine{3. Skaityti is failo}
\DoxyCodeLine{4. Atspausdinti (galutinis pagal nd mediana)}
\DoxyCodeLine{5. Atspausdinti (galutinis pagal nd vidurki)}
\DoxyCodeLine{6. Generuoti studentu faila (v1.0)}
\DoxyCodeLine{7. Skelti studentus i maladec ir L-laivo sarasus, atspausdinti}
\end{DoxyCode}

\item Pasirinkite norimą funkciją ivesdami jos numerį. Input\textquotesingle{}ai apsaugoti tai galima ir pasiaust \+:D
\item Po 1, 2 arba 3 funkcijos įvykdomo programa grįžta į meniu, atnaujinamas saugomų studentų rodiklio skaičius
\item 4, 5, 7 funkcijos neveiks, jei nebus saugomi bent vieno studento duomenys \subsection*{-\/ Po 4, 5, 6 arba 7 funkcijos įvykdymo programa baigia darbą }
\end{DoxyItemize}

\subsection*{Versijų istorija (changelog)}



 \subsubsection*{\href{https://github.com/gitguuddd/Obj_Duomenu_Apdorojimas_class/releases/tag/v1.5}{\texttt{ v1.\+5}}}

(2019-\/05-\/07)

 {\bfseries{Pridėta}}
\begin{DoxyItemize}
\item {\ttfamily \mbox{\hyperlink{class_stud}{Stud}}} derived klasės abstrakti bazinė klasė {\ttfamily \mbox{\hyperlink{class_humie}{Humie}}}
\item {\ttfamily \mbox{\hyperlink{class_important___values}{Important\+\_\+\+Values}}} klasė kurioje saugomi ir apsaugomi seniau per visą programą buvę išmėtyti globalūs kintamieji
\item {\ttfamily \mbox{\hyperlink{class_stud}{Stud}}} klasės $>$$>$ ir $<$$<$ operatoriai
\item {\ttfamily \mbox{\hyperlink{class_stud}{Stud}}} klasės {\ttfamily scan\+\_\+name} ir {\ttfamily scan\+\_\+surname} funkcijos
\item {\ttfamily handle\+\_\+nd} funkcija skirta {\ttfamily ndcount} kintamojo įvedimo į {\ttfamily \mbox{\hyperlink{class_important___values}{Important\+\_\+\+Values}}} tikrinimui
\end{DoxyItemize}

{\bfseries{Koreguota}}
\begin{DoxyItemize}
\item Iš {\ttfamily \mbox{\hyperlink{class_stud}{Stud}}} klasės į {\ttfamily \mbox{\hyperlink{class_humie}{Humie}}} klasę perkelti {\ttfamily m\+\_\+name, m\+\_\+surname, m\+\_\+nd} kintamieji ir su jais susyjusios funkcijos. Taip užtikrinama, kad {\ttfamily \mbox{\hyperlink{class_humie}{Humie}}} klasė saugos įvestus duomenis, o {\ttfamily \mbox{\hyperlink{class_stud}{Stud}}} klasė -\/ išvedimui paruoštus duomenis
\item Panaikinti beveik visi padriki globalūs kintamieji (palikti globalūs vektoriai, dekai, listai ir t.\+t). Minėti kintamieji perkelti į {\ttfamily \mbox{\hyperlink{class_important___values}{Important\+\_\+\+Values}}} klasę
\item Dėl sutikto \char`\"{}circular include\char`\"{} teko pertvarkyt headeriuose include\textquotesingle{}intus kitus headerius \+:(
\item Realizuotų $>$$>$ ir $<$$<$ {\ttfamily \mbox{\hyperlink{class_stud}{Stud}}} klasės operatorių deka buvo pertvarkytos template\textquotesingle{}inė {\ttfamily read\+\_\+file2} , nuskaitymo nuo {\ttfamily kursiokai.\+txt} failo {\ttfamily read\+\_\+file2}, template\textquotesingle{}inė spausdinimo į failą {\ttfamily print\+\_\+to\+\_\+file} funkcijos
\item Į github pridėtas {\ttfamily build} aplankas su jame esančiu kursiokai.\+txt failu
\item Atnaujinta informacija apie programos struktūra
\end{DoxyItemize}

{\bfseries{Žinomos problemos/ką galima pagerinti}}
\begin{DoxyItemize}
\item Vis dar galima patobulinti programos struktūrą
\item Kintamųjų ir funkcijų pavadinimų standartai \+:DD
\end{DoxyItemize}



 \subsubsection*{\href{https://github.com/gitguuddd/Obj_Duomenu_Apdorojimas_class/releases/tag/v1.2}{\texttt{ v1.\+2}}}

(2019-\/05-\/02)



{\bfseries{Pridėta}}
\begin{DoxyItemize}
\item {\ttfamily Rule of five} {\ttfamily \mbox{\hyperlink{class_stud}{Stud}}} klasės operatoriai
\item Įvairūs {\ttfamily \mbox{\hyperlink{class_stud}{Stud}}} klasės palyginimo operatoriai
\end{DoxyItemize}

{\bfseries{Koreguota}}
\begin{DoxyItemize}
\item Programos veikimo laiko skaičiavimas dabar realizuojamas per {\ttfamily \mbox{\hyperlink{_timer_8h}{Timer.\+h}}} klasę
\item \mbox{\hyperlink{class_stud}{Stud}} klasė ir jos funkcijos iš {\ttfamily \mbox{\hyperlink{_mutual_8h}{Mutual.\+h}}} ir {\ttfamily \mbox{\hyperlink{_mutual_8cpp}{Mutual.\+cpp}}} iškeliamos į {\ttfamily \mbox{\hyperlink{_stud_8h}{Stud.\+h}}} ir {\ttfamily \mbox{\hyperlink{_stud_8cpp}{Stud.\+cpp}}}
\item {\ttfamily \mbox{\hyperlink{_stud_8h}{Stud.\+h}}, \mbox{\hyperlink{_stud_8cpp}{Stud.\+cpp}}, \mbox{\hyperlink{_timer_8h}{Timer.\+h}}} failai laikomi classes aplankale
\item Programos metu sugeneruoti failai saugomi {\ttfamily build} aplankale \subsection*{-\/ Dokumentacijoje atnaujinta informacija apie programos struktūrą }
\end{DoxyItemize}

\subsubsection*{\href{https://github.com/gitguuddd/Obj_Duomenu_Apdorojimas_class/releases/tag/v1.1}{\texttt{ v1.\+1}}}

(2019-\/04-\/23)

{\bfseries{Koreguota}}
\begin{DoxyItemize}
\item {\ttfamily \mbox{\hyperlink{class_stud}{Stud}}} struktūra pakeičiama į {\ttfamily \mbox{\hyperlink{class_stud}{Stud}}} klasę
\item Iš klasės narių panaikinti tarpiniai {\ttfamily vid} ir {\ttfamily mvid} kintamieji, nes galutinio balo(vidurkis) ir galutinio balo(mediana) skaičiavimai vyksta setteriuose
\item Pagal {\ttfamily Clang-\/tidy} rekomendacijas keliose vietose panaudotas {\ttfamily std\+::move}
\item Panaikinta daug perteklinių try-\/catch blokų
\end{DoxyItemize}

{\bfseries{Žinomi bugai/ ką galima patobulinti}}
\begin{DoxyItemize}
\item Programos struktūra is still a wreck
\item Kartais {\ttfamily L\+\_\+laivas} S\+TL konteineryje galima išvysti vieną studentą su galutiniu balu, kuris lygus {\ttfamily 5}, galima custom predicate problema
\item Galima patobulinti ir pačią klasę -\/ padaryti ją suprantamesne \subsection*{-\/ Pats metas pradėti taikyti kintamųjų ir funkcijų pavadinimų standartus }
\end{DoxyItemize}

\subsubsection*{\href{https://github.com/gitguuddd/Obj_Duomenu_apdorojimas/releases/tag/v1.0.1}{\texttt{ v1.\+0.\+1}}}

(2019-\/03-\/27)

{\bfseries{Pridėta}}
\begin{DoxyItemize}
\item Funkcijos {\ttfamily t\+\_\+gavo\+\_\+skola, rask\+\_\+\+Minkstus, rask\+\_\+\+Minkstus\+\_\+d, rask\+\_\+\+Kietus, rask\+\_\+\+Kietus\+\_\+d}
\item Papildomos užduoties siūlomi splitinimo variantai pridėti kaip papildoma {\ttfamily d} splitinimo strategija
\item Pridėta galimybė pasirinkti programos veikimo rėžima\+: išsamus/konkretus
\item Skeliant studentus naudojant papildomos užduoties algoritmus skėlimo laikai bus išvedami į {\ttfamily Papildomos\+\_\+uzduoties\+\_\+laikai.\+txt} failą
\item Kai vartotojas pats įveda vardą/pavardę programa tikrina ar varde/pavardėje yra tik raidės, atradųs kitokį simbolį vardas/pavardė yra keičiami į {\ttfamily Bad\+Input}
\end{DoxyItemize}

{\bfseries{Koreguota}}
\begin{DoxyItemize}
\item Sutvarkyti keli \mbox{\hyperlink{_r_e_a_d_m_e_8md}{R\+E\+A\+D\+M\+E.\+md}} failo bug\textquotesingle{}ai
\item Pertvarkyta \mbox{\hyperlink{_r_e_a_d_m_e_8md}{R\+E\+A\+D\+M\+E.\+md}} failo struktūra
\item visi \mbox{\hyperlink{_r_e_a_d_m_e_8md}{R\+E\+A\+D\+M\+E.\+md}} paveiksliukai sukelti į {\ttfamily Memes} folderį
\end{DoxyItemize}



 \subsubsection*{\href{https://github.com/gitguuddd/Obj_Duomenu_apdorojimas/releases/tag/v1.0}{\texttt{ v1.\+0}}}

(2019-\/03-\/25)

{\bfseries{Pridėta}}
\begin{DoxyItemize}
\item Prie naudojamos {\ttfamily c} splitinimo strategijos pridėtos {\ttfamily a} ir {\ttfamily b} strategijos (apie jas -\/ vėliau)
\item Splitinimo strategijos pasirinkimo funkcija {\ttfamily Strat\+\_\+pick}
\end{DoxyItemize}

{\bfseries{Koreguota}}
\begin{DoxyItemize}
\item Sutvarkytas minor bug\textquotesingle{}as\+: išvedant skaičiavimus pagal mediana rašydavo, kad galutinis vertinimas buvo skaičiuotas pagal vidurkį
\item Pasirodo, kad listo nebuvo įmanoma panaudoti su didesniais studentų kiekiais vien dėl mano kaltės, nes neišvalydavau temp struktūros.
\item Paspartintas readinimas iš failo\+: eilutės duomenys nuskaitomi į temp struktūra. Priėjųs eilutės galą, struktūra yra pushbackinama S\+TL konteineryje.
\item Pasikartojantis kodas printinimo, skaičiavimo, vardų analižės vietose pakeistas kreipimusi į template funkcijas {\ttfamily print\+\_\+to\+\_\+file ,calc , names}
\item R\+E\+A\+D\+ME fail\textquotesingle{}e pridėtos naudojimosi, įdiegimo instrukcijos.
\end{DoxyItemize}



 \subsubsection*{\href{https://github.com/gitguuddd/Obj_Duomenu_apdorojimas/releases/tag/v0.5}{\texttt{ v0.\+5}}}

(2019-\/03-\/18)  

{\bfseries{Pridėta}}
\begin{DoxyItemize}
\item {\ttfamily split\+\_\+n\+\_\+print} ir {\ttfamily read\+\_\+file2} funkcijoms sukurti listo ir deque variantai, nes labai pasisekė kurti funkcijų template\textquotesingle{}us (žiūrėti viršuje \+:) )
\item {\ttfamily t\+\_\+\+S\+T\+L\+\_\+pick} funkcija, kuri leidžia vartotojui pasirinkti ar jis norės dirbti su vektoriumi/deku/listu
\item Naujuose inputuose sudėtas input handling\textquotesingle{}as
\item Į benchmark\textquotesingle{}o failą (v0.\+5\+\_\+laikai.\+txt) išvedamas ir pasirinkto S\+TL konteinerio pavadinimas
\item Nustatyta, kad listas lūžta, kai perlipa 19800 saugomų studentų skačių, listo generavimas apribotas iki 10000 studentų.
\item 
\end{DoxyItemize}

{\bfseries{Koreguota}}
\begin{DoxyItemize}
\item Visur sutvarkytas lygiavimas
\item {\ttfamily v0.\+4\+\_\+laikai.\+txt} keičiama į {\ttfamily v0.\+5\+\_\+laikai.\+txt}
\item {\ttfamily Generuoti studentu faila (v0.\+4)} keičiama į {\ttfamily Generuoti studentu faila (v0.\+5)}
\item Pasirodo, kad {\ttfamily Generuoti studentu faila (v0.\+5)} funkcija visados generuodavo vienu studentu mažiau nei reikia. Tai sutvarkyta
\item Namų darbų skaičiaus maksimalus pasirinkimas padidintas iki 1000000
\item Keliose vietose panaikinti nereikalingi try-\/catch blokai
\item Po 0.\+5 studentų generavimo programa į menių negrįžta ir tiesiog baiga darbą.
\end{DoxyItemize}



 \subsubsection*{\href{https://github.com/gitguuddd/Obj_Duomenu_apdorojimas/releases/tag/v0.4}{\texttt{ v0.\+4}}}

(2019-\/03-\/11) 

{\bfseries{Pridėta}}


\begin{DoxyItemize}
\item {\ttfamily Generuoti studentu faila (v0.\+4)} meniu pasirinkimas kuris apjungia šias naujas funkcijas\+:
\item {\bfseries{overload\textquotesingle{}inta {\ttfamily gen\+\_\+student} funkcija}}, kuri sugeneruoja nurodytą kiekį studentų su atsitiktinais vardais, pavardėmis , egzaminų įverčiais ir vartotojo nurodytu namų darbų įverčių kiekiu ir šiuos duomenis išveda į failus su atitinkamais pavadinimais {\ttfamily generuojamu studentu kiekis + studentu.\+txt}
\item {\ttfamily read\+\_\+file2} {\bfseries{funkcija}} kuri nuskaito duomenis iš sugeneruotų failų ir juos sudeda į students vektorių.
\item {\ttfamily split\+\_\+n\+\_\+print} {\bfseries{funkcija}} kuri, pagal vartotojo pasirinkimą (\mbox{[}m\mbox{]}ediana/\mbox{[}v\mbox{]}idurkis) suskaičiuoja studentų vektoriuje saugomų studentų galutinius pažymius pagal namų darbų įverčių medianą/namų darbų įverčių vidurkį, išrikiuoja stundentus pagal galutinį pažymį, randa iteratorių ties kuriuo galutinis pažymys tampa lygus 5.\+0, panaudoja iteratorių skeliant studentų sąrašą į {\ttfamily mldc} studentus ir {\ttfamily L\+\_\+laivas} studentus, juos atspausdina į failus su atitinkamais vardais {\ttfamily studentu skaičius + studentu + v/m + mldc/\+L\+\_\+laivas}
\item Naudojant {\ttfamily high\+\_\+resolution\+\_\+clock} išmatuotas {\ttfamily gen\+\_\+student}, {\ttfamily read\+\_\+file2} ir {\ttfamily split\+\_\+n\+\_\+print} veikimo laikai dirbant su visais studentų sąrašų variantais (10, 100, 1000, 10000, 100000, 1000000), laikai išvedami į {\ttfamily v0.\+4\+\_\+laikai.\+txt} failą
\item Naujose funkcijose pridėtas input/exception handling\textquotesingle{}as
\item Dabar galima skelti ir atspausdinti ir duomenis kurie buvo sugeneruoti/įvesti paties naudotojo arba nuskaityti iš kursiokai.\+txt failo
\end{DoxyItemize}

{\bfseries{Koreguota}}


\begin{DoxyItemize}
\item Kursioko Igno D. patarimu mt19937 seedinimas pakeistas iš random device į {\ttfamily high\+\_\+resolution\+\_\+clock\+::now().time\+\_\+since\+\_\+epoch().count()}, nes pasirodo, kad mano kompiuteris neturi random device \+:(
\item Sutvarkyti keli minor bug\textquotesingle{}ai susyję su informacijos išvedimu konsolėje
\item Pakeistas \mbox{\hyperlink{_r_e_a_d_m_e_8md}{R\+E\+A\+D\+M\+E.\+md}} failas
\end{DoxyItemize}

{\bfseries{Ką artimiausiame release reiktų pataisyti, bet ko tikriausiai nepataisysių}}
\begin{DoxyItemize}
\item Skeliant vartotojo įvestus/sugeneruotus/nuskaitytus duomenis, sudaryti failai ne visados pasižymi puikiu lygiavimu
\item Sugeneruoti studentų failai (dar neperskelti) irgi nėra puikiai išlygiuoti, nes iškart generuojant ir išvedant nėra galimybės surasti maxname ir maxsurname reikšmių
\item Studentų sąrašų generavimas ir skėlimas prikuria labaiiii daug failų.
\item Dėl multiple definition errorų į {\ttfamily \mbox{\hyperlink{_input_8h}{input.\+h}}} ir {\ttfamily \mbox{\hyperlink{_input_8cpp}{input.\+cpp}}} teko perkelti split\+\_\+n\+\_\+print funkciją, nors ji pagal veikimo principą turėtų priklausyti {\ttfamily \mbox{\hyperlink{_output_8h}{output.\+h}}} ir {\ttfamily \mbox{\hyperlink{_output_8cpp}{output.\+cpp}}} failams
\item Per daug kintamųjų, kurie kaip extern\textquotesingle{}ai yra deklaruojami {\ttfamily \mbox{\hyperlink{_mutual_8h}{mutual.\+h}}}
\end{DoxyItemize}

{\bfseries{Laiko matavimo rezultatai ir pavyzdžiai}}
\begin{DoxyItemize}
\item Visi sarašai buvo sugeneruoti naudojant Release profilį ir -\/O3 flagą
\end{DoxyItemize}



 \subsubsection*{\href{https://github.com/gitguuddd/Obj_Duomenu_apdorojimas/releases/tag/v0.3}{\texttt{ v0.\+3}}}

(2019-\/02-\/25)

{\bfseries{Pridėta}}
\begin{DoxyItemize}
\item {\ttfamily \mbox{\hyperlink{_input_8h}{Input.\+h}}, \mbox{\hyperlink{_mutual_8h}{Mutual.\+h}}, \mbox{\hyperlink{_output_8h}{Output.\+h}}, \mbox{\hyperlink{_mutual_8cpp}{Mutual.\+cpp}}, \mbox{\hyperlink{_output_8cpp}{Output.\+cpp}}, \mbox{\hyperlink{_input_8cpp}{Input.\+cpp}}}
\item Input handling (pagrinde {\ttfamily cin.\+fail()})
\item Naudojami {\ttfamily vector.\+reserve()} ir {\ttfamily vector.\+shrink\+\_\+to\+\_\+fit()}
\item Exception handling ({\ttfamily try -\/ catch}) ties {\ttfamily students.\+push\+\_\+back} ir {\ttfamily Stud.\+nd.\+push\+\_\+back}, nes pereita prie vektorių atminties rezervavimo
\item Input handling nuskaitant failą -\/ galima nuskaityti namų darbų pažymius net ir jei jie yra ne skaičiai arba mazesni uz 0/ didesni uz 10 (aptikus -\/ nulinami), jei netinkamas input yra ilgesnis nei 4 simboliai -\/ didelė rizika sugadinti nuskaitytų duomenų tikslumą
\item Nested switch meniu klaidų spausdinimui
\item Programos struktūros sekcija \mbox{\hyperlink{_r_e_a_d_m_e_8md}{R\+E\+A\+D\+M\+E.\+MD}} faile
\item 
\end{DoxyItemize}

{\bfseries{Koreguota}}
\begin{DoxyItemize}
\item Panaikinti goto, taip pat ir easter egg (L\+I\+N\+E\+K\+A\+P\+UT\+:), perdaryta switch logika.
\item Pasirodo, kad senesnių versijų duomenys nebuvo labai tikslūs, duomenys patikslinti. \subsection*{-\/ kursiokai.\+txt failas tikslingai sucorruptintas norint išbandyti input handling\textquotesingle{}a. }
\end{DoxyItemize}

\subsubsection*{\href{https://github.com/gitguuddd/Obj_Duomenu_apdorojimas/releases/tag/v0.2}{\texttt{ v0.\+2}}}

(2019-\/02-\/18)

{\bfseries{Pridėta}}
\begin{DoxyItemize}
\item Kursiokai.\+txt
\item Rūšiavimas pagal studento vardą ir pavardę
\item Nuskaitymas iš failo
\item Galimybė visais būdais gautus duomenis atspausdinti viename output sąraše
\item Primityvus switch\textquotesingle{}o error handling\textquotesingle{}as naudojant goto
\item Easter egg
\end{DoxyItemize}

{\bfseries{Koreguota}}
\begin{DoxyItemize}
\item Pakeistas \mbox{\hyperlink{_r_e_a_d_m_e_8md}{R\+E\+A\+D\+M\+E.\+md}} failas
\end{DoxyItemize}

{\bfseries{Žinomi trūkumai/ką reikia kuo greičiau ištaisyti}}
\begin{DoxyItemize}
\item Goto switche reikia pakeisti į normalų refactoringa dirbant prie v0.\+3
\item Studentų rikiavimas veiks keistokai jei bus naudojami ne vienodo ilgio vardai ir pavardės. \subsection*{-\/ Nemažai vietų trūksta exception/error handling\textquotesingle{}o }
\end{DoxyItemize}

\subsubsection*{\href{https://github.com/gitguuddd/Obj_Duomenu_apdorojimas/releases/tag/v0.1}{\texttt{ v0.\+1}}}

(2019-\/02-\/18)

{\bfseries{Pridėta}}
\begin{DoxyItemize}
\item C\+Make failas
\item Pirmininis programos {\ttfamily cpp} failas
\item Alternatyvios programos (su masyvu ) {\ttfamily cpp} failas (veikiantis bardakas)
\end{DoxyItemize}

{\bfseries{Koreguota}} \subsection*{-\/ Pakeistas \mbox{\hyperlink{_r_e_a_d_m_e_8md}{R\+E\+A\+D\+M\+E.\+md}} failas }

\subsection*{Benchmark\textquotesingle{}ai}

{\bfseries{Truputis informacijos norint suprasti benchmarkus}}
\begin{DoxyItemize}
\item Visi testai yra vykdomi Release profilyje
\item Visi testai vykdomi su 10 namų darbų.
\item Egzistuoja keturios skėlimo strategijos {\ttfamily A, B, C, D}
\item Strategija {\ttfamily A} -\/ Studentai, kurių galutinis pažymys yra $>$= 5.\+0 yra perkeliami į {\ttfamily mldc} S\+TL konteinerį, kurių yra mažesnis už 5.\+0 -\/ į {\ttfamily L\+\_\+laivas} S\+TL konteinerį. Studentai iš pirminio S\+TL konteinerio netrinami
\item Strategija {\ttfamily B} -\/ Studentai, kurių galutinis pažymys yra $>$=5.\+0 yra perkeliami į {\ttfamily mldc\+\_\+students} S\+TL konteinerį, iš pirminio studentų S\+TL konteinerio šie studentai yra pašalinami, pirminis S\+TL\textquotesingle{}as tampa {\ttfamily L\+\_\+laivas} S\+TL konteineriu
\item Strategija {\ttfamily C} -\/ Pirminis studentų S\+TL konteineris yra išrikiuojamas didėjimo tvarka pagal galutinį pažymį, naudojant {\ttfamily std\+::upper\+\_\+bound} randamas iteratorių {\ttfamily up}, ties kuriuo reikšmės perlipa {\ttfamily 4.\+999999999} ribą. Šis iteratoriaus yra naudojamas {\ttfamily mldc\+\_\+students} S\+TL konteinerio kontruktoriuje, resize{\ttfamily inant pirminį S\+TL konteinerį yra gaunamas}{\ttfamily L\+\_\+laivas}{\ttfamily S\+TL konteineris.}
\item {\ttfamily Strategija}{\ttfamily D}{\ttfamily -\/ Papildomos užduoties siūlomi skėlimo algoritmai\+:}{\ttfamily m}{\ttfamily algoritmas, kuris skolininkus perkelia į atskirą S\+TL konteinerį, šiuos studentus ištrina iš pirminio S\+TL konteinerio per}{\ttfamily erase}{\ttfamily funkciją,}{\ttfamily k}{\ttfamily algoritmas, kuris skolininkus irgi perkelia į atskirą S\+TL konteinerį, kietus studentus perkelia į pirminio S\+TL konteinerio priekį ir naudojant}{\ttfamily resize}{\ttfamily ir}{\ttfamily shrink\+\_\+to\+\_\+fit}{\ttfamily funkcijas iš šio S\+TL konteinerio ištrina perteklinius kietus studentus/skolininkus ir atlaisvina atmintį -\/}{\ttfamily vid}{\ttfamily ir}{\ttfamily med}\`{} atitinkamai reiškia, kad teste galutinis pažymys buvo skaičiuotas pagal namų darbų įverčių vidurkį arba medianą
\end{DoxyItemize}

{\bfseries{v1.\+1}}

{\bfseries{Programos veikimo laikų palyginimas naudojant {\ttfamily \mbox{\hyperlink{class_stud}{Stud}}} struktūra arba {\ttfamily \mbox{\hyperlink{class_stud}{Stud}}} klasę}}


\begin{DoxyItemize}
\item Šiam ir sekantiems ($>$=v1.\+1) sparčiausiems testams bus naudojama {\ttfamily A} dalijimo strategija
\end{DoxyItemize}

\tabulinesep=1mm
\begin{longtabu}spread 0pt [c]{*{3}{|X[-1]}|}
\hline
\PBS\centering \cellcolor{\tableheadbgcolor}\textbf{ }&\PBS\centering \cellcolor{\tableheadbgcolor}\textbf{ {\ttfamily vector$<$struct \mbox{\hyperlink{class_stud}{Stud}}$>$}  }&\PBS\centering \cellcolor{\tableheadbgcolor}\textbf{ {\ttfamily vector$<$class \mbox{\hyperlink{class_stud}{Stud}}$>$}   }\\\cline{1-3}
\endfirsthead
\hline
\endfoot
\hline
\PBS\centering \cellcolor{\tableheadbgcolor}\textbf{ }&\PBS\centering \cellcolor{\tableheadbgcolor}\textbf{ {\ttfamily vector$<$struct \mbox{\hyperlink{class_stud}{Stud}}$>$}  }&\PBS\centering \cellcolor{\tableheadbgcolor}\textbf{ {\ttfamily vector$<$class \mbox{\hyperlink{class_stud}{Stud}}$>$}   }\\\cline{1-3}
\endhead
100000 v  &3.\+94281 s.  &1.\+73632 s.   \\\cline{1-3}
1000000 v  &39.\+8862 s.  &35.\+73931 s.   \\\cline{1-3}
\end{longtabu}



\begin{DoxyItemize}
\item Iš lentelės duomenų matome, kad {\ttfamily \mbox{\hyperlink{class_stud}{Stud}}} realizacija per klasę sugebėjo paspartinti A strategiją.
\end{DoxyItemize}

{\bfseries{Programos veikimo laikų palyginimas naudojant skirtingus optimizavimo flag\textquotesingle{}us}}

\tabulinesep=1mm
\begin{longtabu}spread 0pt [c]{*{4}{|X[-1]}|}
\hline
\PBS\centering \cellcolor{\tableheadbgcolor}\textbf{ }&\PBS\centering \cellcolor{\tableheadbgcolor}\textbf{ -\/O1  }&\PBS\centering \cellcolor{\tableheadbgcolor}\textbf{ -\/O2  }&\PBS\centering \cellcolor{\tableheadbgcolor}\textbf{ -\/O3   }\\\cline{1-4}
\endfirsthead
\hline
\endfoot
\hline
\PBS\centering \cellcolor{\tableheadbgcolor}\textbf{ }&\PBS\centering \cellcolor{\tableheadbgcolor}\textbf{ -\/O1  }&\PBS\centering \cellcolor{\tableheadbgcolor}\textbf{ -\/O2  }&\PBS\centering \cellcolor{\tableheadbgcolor}\textbf{ -\/O3   }\\\cline{1-4}
\endhead
100000 v  &3.\+799484 s.  &3.\+532669 s.  &3.\+595513 s.   \\\cline{1-4}
1000000 v  &41.\+12474 s.  &40.\+11205 s.  &34.\+53586 s.   \\\cline{1-4}
\end{longtabu}



\begin{DoxyItemize}
\item Kaip ir reikėjo tikėtis -\/ O3 optimizacijos flag\textquotesingle{}as labiausiai padėjo programai. Neaiškumą lyginant 100000v -\/O2 ir 100000v -\/O3 galima paaiškinti tuo, kad duomenų tikslumui koją galėjo pakišti atsitiktinis kompiuteriuo apkrovos kitimas.
\item 
\end{DoxyItemize}

{\bfseries{v1.\+0.\+1}}

{\bfseries{Papidomos užduoties skėlimo algoritmų (D strategijos) laikai dirbant su vector arba deque}}

\tabulinesep=1mm
\begin{longtabu}spread 0pt [c]{*{3}{|X[-1]}|}
\hline
\PBS\centering \cellcolor{\tableheadbgcolor}\textbf{ S\+TL konteineris  }&\PBS\centering \cellcolor{\tableheadbgcolor}\textbf{ std\+::vector$<$$>$  }&\PBS\centering \cellcolor{\tableheadbgcolor}\textbf{ std\+::deque$<$$>$   }\\\cline{1-3}
\endfirsthead
\hline
\endfoot
\hline
\PBS\centering \cellcolor{\tableheadbgcolor}\textbf{ S\+TL konteineris  }&\PBS\centering \cellcolor{\tableheadbgcolor}\textbf{ std\+::vector$<$$>$  }&\PBS\centering \cellcolor{\tableheadbgcolor}\textbf{ std\+::deque$<$$>$   }\\\cline{1-3}
\endhead
10000 m  &4.\+71733 s.  &1.\+65675 s.   \\\cline{1-3}
10000 k  &11.\+7855 s.  &0.\+004923 s.   \\\cline{1-3}
100000 m  &497.\+472 s.  &175.\+256 s.   \\\cline{1-3}
100000 k  &3144.\+7 s.  &0.\+061832 s.   \\\cline{1-3}
1000000 m  &inf.  &inf.   \\\cline{1-3}
1000000 k  &inf.  &0.\+600361 s.   \\\cline{1-3}
\end{longtabu}



\begin{DoxyItemize}
\item Iš lentelės duomenų matome, kad {\ttfamily m} algoritmas vektoriui buvo palankesnis nei {\ttfamily k} algoritmas.\+Tai galima paaiškinti tuo, kad {\ttfamily k} algoritme naudojamas kėlimas į priekį vektoriui yra O(\+N) sudėtingumo funkcija ir gerokai apsunkina vektoriaus darba -\/ labai daug laiko užima saugomų objektų perstumdymas.
\item Tačiau deque atveju {\ttfamily k} algoritmas buvo totalus game winner, nes deque turi {\ttfamily push\+\_\+front} funkciją ir ją labai sėkmingai panaudoja {\ttfamily k} algoritmą.\+Tai yra vienintelė kombinacija, kuri pabaigė darbą su 1000000 studentų.
\end{DoxyItemize}

{\bfseries{v1.\+0}}

{\bfseries{Skirtingų S\+TL konteineriu naudojamos atminties kiekis pritaikant a strategiją}}

\tabulinesep=1mm
\begin{longtabu}spread 0pt [c]{*{4}{|X[-1]}|}
\hline
\PBS\centering \cellcolor{\tableheadbgcolor}\textbf{ S\+TL konteineris  }&\PBS\centering \cellcolor{\tableheadbgcolor}\textbf{ std\+::vector$<$$>$  }&\PBS\centering \cellcolor{\tableheadbgcolor}\textbf{ std\+::deque$<$$>$  }&\PBS\centering \cellcolor{\tableheadbgcolor}\textbf{ std\+::list$<$$>$   }\\\cline{1-4}
\endfirsthead
\hline
\endfoot
\hline
\PBS\centering \cellcolor{\tableheadbgcolor}\textbf{ S\+TL konteineris  }&\PBS\centering \cellcolor{\tableheadbgcolor}\textbf{ std\+::vector$<$$>$  }&\PBS\centering \cellcolor{\tableheadbgcolor}\textbf{ std\+::deque$<$$>$  }&\PBS\centering \cellcolor{\tableheadbgcolor}\textbf{ std\+::list$<$$>$   }\\\cline{1-4}
\endhead
Naudojamos atminties kiekis programos pradžioje  &\PBS\centering 635 MB  &\PBS\centering 966.\+6 MB  &\PBS\centering 971.\+6 MB   \\\cline{1-4}
10 skirtumas  &\PBS\centering + 0.\+2 MB  &\PBS\centering + 0.\+2 MB  &\PBS\centering + 0.\+3 MB   \\\cline{1-4}
100 skirtumas  &\PBS\centering + 0.\+3 MB  &\PBS\centering + 0.\+4 MB  &\PBS\centering + 0.\+7 MB   \\\cline{1-4}
1000 skirtumas  &\PBS\centering + 0.\+9 MB  &\PBS\centering + 1.\+8 MB  &\PBS\centering + 0.\+9 MB   \\\cline{1-4}
10000 skirtumas  &\PBS\centering + 3.\+2 MB  &\PBS\centering + 3.\+9 MB  &\PBS\centering + 2.\+4 MB   \\\cline{1-4}
100000 skirtumas  &\PBS\centering + 34.\+1 MB  &\PBS\centering + 30.\+8 MB  &\PBS\centering + 36.\+9 MB   \\\cline{1-4}
1000000 skirtumas  &\PBS\centering + 616.\+2 MB  &\PBS\centering + 293.\+4 MB  &\PBS\centering + 322.\+6 MB   \\\cline{1-4}
\end{longtabu}

\begin{DoxyItemize}
\item Iš lentelės duomenų matome, kad {\ttfamily vector} S\+TL konteineris yra labai jautrus tokiam atminties švaistymui
\end{DoxyItemize}

{\bfseries{Programos benchmark\textquotesingle{}as prieš optimizuojant a ir b strategijas}}
\begin{DoxyItemize}
\item a ir b strategijoms buvo būdingas štai toks kodas 
\begin{DoxyCode}{0}
\DoxyCodeLine{a strategija}
\DoxyCodeLine{}
\DoxyCodeLine{if(pchoice=='v')}
\DoxyCodeLine{            for (Stud\& Stud : students\_d)}
\DoxyCodeLine{                (Stud.vid2>=test.vid2)?mldc\_d.push\_back(Stud):L\_laivsd.push\_back(Stud);}
\DoxyCodeLine{        else if(pchoice=='m')}
\DoxyCodeLine{            for (Stud\& Stud : students\_d)}
\DoxyCodeLine{                (Stud.mvid>=test.mvid)?mldc\_d.push\_back(Stud):L\_laivsd.push\_back(Stud);\}}
\DoxyCodeLine{}
\DoxyCodeLine{b strategija}
\DoxyCodeLine{}
\DoxyCodeLine{if(pchoice=='v')\{}
\DoxyCodeLine{            std::copy\_if(students\_d.begin(), students\_d.end(), std::back\_inserter(mldc\_students\_d) ,[](auto v) \{return v.vid2>test.vid2;\});}
\DoxyCodeLine{            students\_d.erase(std::remove\_if(students\_d.begin(),students\_d.end(),[](auto v) \{return v.vid2>test.vid2;\}),students\_d.end());}
\DoxyCodeLine{        \}}
\DoxyCodeLine{        if (pchoice=='m')\{}
\DoxyCodeLine{            std::copy\_if(students\_d.begin(), students\_d.end(), std::back\_inserter(mldc\_students\_d) ,[](auto v) \{return v.mvid>test.mvid;\});}
\DoxyCodeLine{            students\_d.erase(std::remove\_if(students\_d.begin(),students\_d.end(),[](auto v) \{return v.mvid>test.mvid;\}),students\_d.end());}
\DoxyCodeLine{        \}}
\end{DoxyCode}
 \tabulinesep=1mm
\begin{longtabu}spread 0pt [c]{*{4}{|X[-1]}|}
\hline
\PBS\centering \cellcolor{\tableheadbgcolor}\textbf{ S\+TL konteineris  }&\PBS\centering \cellcolor{\tableheadbgcolor}\textbf{ std\+::vector$<$$>$  }&\PBS\centering \cellcolor{\tableheadbgcolor}\textbf{ std\+::deque$<$$>$  }&\PBS\centering \cellcolor{\tableheadbgcolor}\textbf{ std\+::list$<$$>$   }\\\cline{1-4}
\endfirsthead
\hline
\endfoot
\hline
\PBS\centering \cellcolor{\tableheadbgcolor}\textbf{ S\+TL konteineris  }&\PBS\centering \cellcolor{\tableheadbgcolor}\textbf{ std\+::vector$<$$>$  }&\PBS\centering \cellcolor{\tableheadbgcolor}\textbf{ std\+::deque$<$$>$  }&\PBS\centering \cellcolor{\tableheadbgcolor}\textbf{ std\+::list$<$$>$   }\\\cline{1-4}
\endhead
10med. A  &0.\+004149 s.  &0.\+007609 s.  &0.\+004057 s.   \\\cline{1-4}
10vid. A  &0.\+004931 s.  &0.\+004841 s.  &0.\+004829 s.   \\\cline{1-4}
10med. B  &0.\+006466 s.  &0.\+008885 s.  &0.\+005813 s.   \\\cline{1-4}
10vid. B  &0.\+00897 s.  &0.\+006701 s.  &0.\+006041 s.   \\\cline{1-4}
10med. C  &0.\+004168 s.  &0.\+009851 s.  &0.\+005049 s.   \\\cline{1-4}
10vid .C  &0.\+003733 s.  &0.\+003545 s.  &0.\+004209 s.   \\\cline{1-4}
100med. A  &0.\+008164 s.  &0.\+008695 s.  &0.\+007641 s.   \\\cline{1-4}
100vid. A  &0.\+006982 s.  &0.\+008061 s.  &0.\+007104 s.   \\\cline{1-4}
100med. B  &0.\+008847 s.  &0.\+00708 s.  &0.\+007023 s.   \\\cline{1-4}
100vid. B  &0.\+010152 s.  &0.\+10384 s.  &0.\+006812 s.   \\\cline{1-4}
100med. C  &0.\+008884 s.  &0.\+007204 s.  &0.\+009949 s.   \\\cline{1-4}
100vid. C  &0.\+009582 s.  &0.\+009121 s.  &0.\+009793 s.   \\\cline{1-4}
1000med. A  &0.\+045407 s.  &0.\+030931 s.  &0.\+045012 s.   \\\cline{1-4}
1000vid. A  &0.\+035778 s.  &0.\+031794 s.  &0.\+038341 s.   \\\cline{1-4}
1000med. B  &0.\+033443 s.  &0.\+034769 s.  &0.\+029803 s.   \\\cline{1-4}
1000vid. B  &0.\+034354 s.  &0.\+039826 s.  &0.\+03594 s.   \\\cline{1-4}
1000med. C  &0.\+03992 s.  &0.\+0331154 s.  &0.\+039117 s.   \\\cline{1-4}
1000vid. C  &0.\+054393 s.  &0.\+034342 s.  &0.\+032307 s.   \\\cline{1-4}
10000med. A  &0.\+324686 s.  &0.\+256643 s.  &0.\+263765 s.   \\\cline{1-4}
10000vid. A  &0.\+289927 s.  &0.\+25493 s.  &0.\+276006 s.   \\\cline{1-4}
10000med. B  &0.\+320333 s.  &0.\+260436 s.  &0.\+280009 s.   \\\cline{1-4}
10000vid. B  &0.\+263332 s.  &0.\+323455 s.  &0.\+288229 s.   \\\cline{1-4}
10000med. C  &0.\+411803 s.  &0.\+298277 s.  &0.\+309161 s.   \\\cline{1-4}
10000vid. C  &0.\+38127 s.  &1.\+135304 s.  &0.\+349428 s.   \\\cline{1-4}
100000med. A  &3.\+065104 s.  &2.\+835945 s.  &3.\+072419 s.   \\\cline{1-4}
100000vid. A  &1.\+805628 s.  &2.\+696151 s.  &3.\+472723 s.   \\\cline{1-4}
100000med. B  &4.\+301535 s.  &2.\+959285 s.  &2.\+5732259 s.   \\\cline{1-4}
100000vid. B  &3.\+135811 s.  &3.\+096747 s.  &3.\+225552 s.   \\\cline{1-4}
100000med. C  &4.\+298746 s.  &2.\+977456 s.  &3.\+801098 s.   \\\cline{1-4}
100000vid. C  &3.\+871526 s.  &2.\+391564 s.  &3.\+627411 s.   \\\cline{1-4}
1000000med. A  &29.\+86407 s.  &27.\+74826 s.  &30.\+25982 s.   \\\cline{1-4}
1000000vid. A  &34.\+03236 s.  &28.\+82433 s.  &31.\+97502 s.   \\\cline{1-4}
1000000med. B  &32.\+57199 s.  &30.\+12409 s.  &31.\+86533 s.   \\\cline{1-4}
1000000vid. B  &31.\+42009 s.  &32.\+16509 s.  &30.\+41746 s.   \\\cline{1-4}
1000000med. C  &40.\+5757 s.  &29.\+35842 s.  &35.\+62856 s.   \\\cline{1-4}
1000000vid. C  &38.\+07228 s.  &38.\+83359 s.  &37.\+20177 s.   \\\cline{1-4}
\end{longtabu}

\item Iš lentelės duomenų matome, kad mano programos atveju dirbant su skirtingais S\+TL konteineriais programa sugaišta panašų kiekį laiko
\end{DoxyItemize}

{\bfseries{Programos darbas po a ir b strategijų optimizavimo}}
\begin{DoxyItemize}
\item {\ttfamily c} strategija į lentelę neįtraukta, nes ji nesikeitė. Optimizavus {\ttfamily b} strategiją, savo veikimo principu ji tapo labai panaši į {\ttfamily c} strategiją
\end{DoxyItemize}


\begin{DoxyCode}{0}
\DoxyCodeLine{a ir b strategijos}
\DoxyCodeLine{}
\DoxyCodeLine{if(pchoice=='v')}
\DoxyCodeLine{                up\_l=stable\_partition(students\_l.begin(),students\_l.end(),[](auto v) \{return v.vid2<test.vid2;\});\}}
\DoxyCodeLine{            else if(pchoice=='m')}
\DoxyCodeLine{                up\_l=stable\_partition(students\_l.begin(),students\_l.end(),[](auto v) \{return v.mvid<test.mvid;\});}
\end{DoxyCode}

\begin{DoxyItemize}
\item Šiuo atveju rastas {\ttfamily up\+\_\+l} iteratorius {\ttfamily a} strategijos atveju yra panaudojamas konstruojant {\ttfamily L\+\_\+laivasl} ir {\ttfamily mldc\+\_\+l} listus, {\ttfamily b} strategijos atveju šis iteratorius panaujomas konstruojant {\ttfamily mldc\+\_\+students\+\_\+l} list\textquotesingle{}ą
\end{DoxyItemize}

\tabulinesep=1mm
\begin{longtabu}spread 0pt [c]{*{4}{|X[-1]}|}
\hline
\PBS\centering \cellcolor{\tableheadbgcolor}\textbf{ S\+TL konteineris  }&\PBS\centering \cellcolor{\tableheadbgcolor}\textbf{ std\+::vector$<$$>$  }&\PBS\centering \cellcolor{\tableheadbgcolor}\textbf{ std\+::deque$<$$>$  }&\PBS\centering \cellcolor{\tableheadbgcolor}\textbf{ std\+::list$<$$>$   }\\\cline{1-4}
\endfirsthead
\hline
\endfoot
\hline
\PBS\centering \cellcolor{\tableheadbgcolor}\textbf{ S\+TL konteineris  }&\PBS\centering \cellcolor{\tableheadbgcolor}\textbf{ std\+::vector$<$$>$  }&\PBS\centering \cellcolor{\tableheadbgcolor}\textbf{ std\+::deque$<$$>$  }&\PBS\centering \cellcolor{\tableheadbgcolor}\textbf{ std\+::list$<$$>$   }\\\cline{1-4}
\endhead
10med. A  &0.\+004988 s.  &0.\+007974 s.  &0.\+0039868 s.   \\\cline{1-4}
10vid. A  &0.\+005936 s.  &0.\+002994 s.  &0.\+002991 s.   \\\cline{1-4}
10med. B  &0.\+004985 s.  &0.\+005987 s.  &0.\+003991 s.   \\\cline{1-4}
10vid. B  &0.\+004985 s.  &0.\+003962 s.  &0.\+0049816 s.   \\\cline{1-4}
100med. A  &0.\+011115 s.  &0.\+008017 s.  &0.\+005984 s.   \\\cline{1-4}
100vid. A  &0.\+009008 s.  &0.\+005985 s.  &0.\+00798 s.   \\\cline{1-4}
100med. B  &0.\+006915 s.  &0.\+007979 s.  &0.\+008972 s.   \\\cline{1-4}
100vid. B  &0.\+007977 s.  &0.\+007942 s.  &0.\+008011 s.   \\\cline{1-4}
1000med. A  &0.\+031731 s.  &0.\+03291 s.  &0.\+034002 s.   \\\cline{1-4}
1000vid. A  &0.\+029913 s.  &0.\+037914 s.  &0.\+030915 s.   \\\cline{1-4}
1000med. B  &0.\+030947 s.  &0.\+034943 s.  &0.\+032974 s.   \\\cline{1-4}
1000vid. B  &0.\+031914 s.  &0.\+032948 s.  &0.\+030878 s.   \\\cline{1-4}
10000med. A  &0.\+260302 s.  &0.\+317148 s.  &0.\+2702790 s.   \\\cline{1-4}
10000vid. A  &0.\+291251 s.  &0.\+277272 s.  &0.\+278252 s.   \\\cline{1-4}
10000med. B  &0.\+266288 s.  &0.\+302154 s.  &0.\+274262 s.   \\\cline{1-4}
10000vid. B  &0.\+281282 s.  &0.\+268279 s.  &0.\+279253 s.   \\\cline{1-4}
100000med. A  &2.\+902221 s.  &3.\+120653 s.  &3.\+318164 s.   \\\cline{1-4}
100000vid. A  &3.\+013925 s.  &3.\+081752 s.  &3.\+280935 s.   \\\cline{1-4}
100000med. B  &2.\+917195 s.  &2.\+934117 s.  &3.\+274237 s.   \\\cline{1-4}
100000vid. B  &3.\+018921 s.  &3.\+047839 s.  &2.\+751651 s.   \\\cline{1-4}
1000000med. A  &28.\+70126 s.  &29.\+83323 s.  &29.\+43332 s.   \\\cline{1-4}
1000000vid. A  &29.\+14299 s.  &30.\+59215 s.  &29.\+58191 s.   \\\cline{1-4}
1000000med. B  &29.\+73743 s.  &29.\+01442 s.  &29.\+41837 s.   \\\cline{1-4}
1000000vid. B  &28.\+89283 s.  &30.\+00679 s.  &29.\+0454 s.   \\\cline{1-4}
\end{longtabu}


{\bfseries{v0.\+5}} {\bfseries{Testavimo rezultatai naudojant skirtingus S\+TL konteinerius}}

\tabulinesep=1mm
\begin{longtabu}spread 0pt [c]{*{4}{|X[-1]}|}
\hline
\PBS\centering \cellcolor{\tableheadbgcolor}\textbf{ S\+TL konteineris  }&\PBS\centering \cellcolor{\tableheadbgcolor}\textbf{ std\+::vector$<$$>$  }&\PBS\centering \cellcolor{\tableheadbgcolor}\textbf{ std\+::deque$<$$>$  }&\PBS\centering \cellcolor{\tableheadbgcolor}\textbf{ std\+::list$<$$>$   }\\\cline{1-4}
\endfirsthead
\hline
\endfoot
\hline
\PBS\centering \cellcolor{\tableheadbgcolor}\textbf{ S\+TL konteineris  }&\PBS\centering \cellcolor{\tableheadbgcolor}\textbf{ std\+::vector$<$$>$  }&\PBS\centering \cellcolor{\tableheadbgcolor}\textbf{ std\+::deque$<$$>$  }&\PBS\centering \cellcolor{\tableheadbgcolor}\textbf{ std\+::list$<$$>$   }\\\cline{1-4}
\endhead
10vid  &0.\+009014 s.  &0.\+005128 s.  &0.\+003992 s.   \\\cline{1-4}
10med  &0.\+006949 s.  &0.\+002992 s.  &0.\+004979 s.   \\\cline{1-4}
100vid  &0.\+007017 s.  &0.\+009007 s.  &0.\+00698 s.   \\\cline{1-4}
100med  &0.\+007977 s.  &0.\+007982 s.  &0.\+008977 s.   \\\cline{1-4}
1000vid  &0.\+035963 s.  &0.\+040521 s.  &0.\+4627448 s.   \\\cline{1-4}
1000med  &0.\+031945 s.  &0.\+032907 s.  &0.\+191491 s.   \\\cline{1-4}
10000vid  &0.\+328088 s.  &0.\+323124 s.  &7.\+471561 s.   \\\cline{1-4}
10000med  &0.\+310168 s.  &0.\+308523 s.  &19.\+684555 s.   \\\cline{1-4}
100000vid  &3.\+757647 s.  &3.\+352958 s.  &Nėra   \\\cline{1-4}
100000med  &3.\+35035 s.  &3.\+537462 s.  &Nėra   \\\cline{1-4}
1000000vid  &37.\+74918 s.  &42.\+60728 s.  &Nėra   \\\cline{1-4}
1000000med  &41.\+558277 s.  &37.\+02453 s.  &Nėra   \\\cline{1-4}
\end{longtabu}


{\bfseries{v0.\+4}}
\begin{DoxyItemize}
\item {\bfseries{Laiko matavimo rezultatų failo vaizdas visuose sąrašuose generuojant po 10 namų darbų pažymių ir pasirinkus galutinio pažymio skaičiavimą pagal vidurkį}} 
\item {\bfseries{Laiko matavimo rezultatų failo vaizdas visuose sąrašuose generuojant po 10 namų darbų pažymių ir pasirinkus galutinio pažymio skaičiavimą pagal medianą}} !\mbox{[}alt text\mbox{]}(./\+Memes/med10.png \char`\"{}\+:)\char`\"{})
\end{DoxyItemize}



 \subsection*{Programos struktūra}


\begin{DoxyItemize}
\item {\ttfamily \mbox{\hyperlink{_input_8h}{Input.\+h}}} ir {\ttfamily \mbox{\hyperlink{_input_8cpp}{Input.\+cpp}}} funkcijos/kintamieji/include\textquotesingle{}ai susyję su duomenų įvedimu
\item {\ttfamily \mbox{\hyperlink{_output_8h}{Output.\+h}}}ir {\ttfamily \mbox{\hyperlink{_output_8cpp}{Output.\+cpp}}} funkcijos/kintamieji/include\textquotesingle{}ai susyję su duomenų išvedimu
\item {\ttfamily \mbox{\hyperlink{_mutual_8h}{Mutual.\+h}}}ir {\ttfamily \mbox{\hyperlink{_mutual_8cpp}{Mutual.\+cpp}}} funkcijos/kintamieji/include\textquotesingle{}ai bendri visai programai ({\ttfamily \#include iostream},{\ttfamily \mbox{\hyperlink{_mutual_8h_af5dbaf849b333f367635716ee55d1e3d}{handle\+\_\+input()}}} )
\item {\ttfamily \mbox{\hyperlink{main_8cpp}{main.\+cpp}}} pagrindinis failas -\/ meniu
\item {\ttfamily \mbox{\hyperlink{_stud_8h}{Stud.\+h}}} ir {\ttfamily \mbox{\hyperlink{_stud_8cpp}{Stud.\+cpp}}} \mbox{\hyperlink{class_stud}{Stud}} klasei reikalingi include{\ttfamily ai ir funkcijos -\/}{\ttfamily \mbox{\hyperlink{_important___values_8h}{Important\+\_\+\+Values.\+h}}}{\ttfamily ir}{\ttfamily \mbox{\hyperlink{_important___values_8cpp}{Important\+\_\+\+Values.\+cpp}}}{\ttfamily sklandų programos veikimą užtikrinantys kintamieji ir su jais dirbančios funkcijos -\/}{\ttfamily \mbox{\hyperlink{_humie_8h}{Humie.\+h}}}{\ttfamily ir}{\ttfamily \mbox{\hyperlink{_humie_8cpp}{Humie.\+cpp}}}{\ttfamily \mbox{\hyperlink{class_humie}{Humie}} klasei reikalingi include}ai ir funkcijos
\item {\ttfamily \mbox{\hyperlink{_timer_8h}{Timer.\+h}}} timer klasė 
\end{DoxyItemize}